% slides.tex
\documentclass[20pt]{beamer}
\usepackage{listings}
\usepackage[utf8]{inputenc}
\usepackage{color}
\usepackage{graphicx}

\usetheme{default}
\usecolortheme{dove}
\useoutertheme{default}

% Slightly smaller title
\setbeamerfont{frametitle}{size=\large}
\setbeamerfont{verb}{size=\small}

% lst settings
\lstset{
    language=Haskell,
    basicstyle=\small,
    gobble=4
}

\newcommand{\vspaced}{
    \vspace{5mm}
}

\begin{document}

\title{WebSockets}
\subtitle{GhentFPG}
\author{Jasper Van der Jeugt}
\date{October 4, 2011}

\begin{frame}[plain]
    \titlepage
\end{frame}

\begin{frame}{WebSockets}
    Bidirectional, real-time communication between the browser and the server
\end{frame}

\begin{frame}{WebSockets}
    Chatting is allowed during this talk: \\
    \small{\texttt{jaspervdj.be/websockets-example}} \\
    \vspaced
    Works in: Firefox $\geq 6$, Chromium $\geq 14$
\end{frame}

\begin{frame}[fragile]{Client-side: JavaScript}
    \begin{lstlisting}
    var ws = new WebSocket(uri);

    ws.onmessage = function(event) {
        alert(event.data);
    }

    ws.onopen = function() {
        ws.send('Hello, server.');
    }
    \end{lstlisting}
\end{frame}

\begin{frame}{Server-side: WebSockets library}
    Built around enumerator/iteratee \\
    Safe asynchronous sending \\
    Integrates with existing frameworks \\
    Easy to learn and use \\
\end{frame}

\begin{frame}[plain]
    Let's look at some code!
\end{frame}

\end{document}
